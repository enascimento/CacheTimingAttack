\documentclass[9pt,handout]{beamer}
\usepackage{beamerthemesplit}
\usepackage{pgfpages}
\usepackage{verbatim}
\usepackage{fancybox}
\usepackage{color}

\usepackage{algorithm}
%\usepackage{algorithmic}
\usepackage{amsmath}
\usepackage{amsthm}
\usepackage{algpseudocode}
\usepackage{algorithmicx}% http://ctan.org/pkg/algorithmicx
\usepackage{lipsum}% http://ctan.org/pkg/lipsum
\usepackage{xifthen}% http://ctan.org/pkg/xifthen
\usepackage{needspace}% http://ctan.org/pkg/needspace
\usepackage{hyperref}% http://ctan.org/pkg/hyperref

% ================ ALGORITHM ENVIRONMENT ================
\newcounter{numberedAlg}% Algorithm counter
\newenvironment{numberedAlg}[1][]%
  {% \begin{numberedAlg}[#1]
    \needspace{2\baselineskip}% At least 2\baselineskip required, otherwise break
    \noindent \rule{\linewidth}{1pt} \endgraf% Top rule
    \refstepcounter{numberedAlg}% For correct reference of algorithm
    \centering \textsc{Algorithm}~\thenumberedAlg%
    \ifthenelse{\isempty{#1}}{}{:\ #1}% Typeset name (if provided)
  }{% \end{numberedAlg}
  \noindent \rule{\linewidth}{1pt}% Bottom rule
  }%

\usetheme{Antibes}
\usecolortheme{beaver}
\title[AES Timing Attacks]{AES Timing Attacks}

\usepackage{mathptmx}
\usepackage[scaled=.90]{helvet}
\usepackage{courier}
\usepackage[T1]{fontenc}

%\pgfpagesuselayout{4 on 1}[letterpaper,border shrink=5mm]

\institute[RIT]{}
\date{\today}
%\subtitle{}
\author{Hardware and Software Design for Cryptographic Applications}
%\institute[]{}
\date{\today}
\begin{document}

%%%%%
%%
%% Resource link: http://www.math-linux.com/spip.php?article77
%%
%%%%

\begin{frame}
	\titlepage
\end{frame}

\section{Side Channel Attacks}
\begin{frame}
	\frametitle{Ciphers as a Black Box}
In theory, encryption (and decryption) implementations operate as black boxes.
\begin{figure}
\centering
\includegraphics[scale = 0.5]{images/blackbox.pdf}
\end{figure}

\end{frame}

\begin{frame}
	\frametitle{Information Leakage}
In reality, it's hard to prevent additional information from being leaked at runtime.
\begin{figure}
\centering
\includegraphics[scale = 0.5]{images/leaked.pdf}
\end{figure}
	
\end{frame}

\begin{frame}
	\frametitle{Side Channel Attacks}
	\textbf{Definition}: Any attack on a cryptosystem using information leaked given off as a byproduct of the physical implementation of the cryptosystem, rather than a theoretical weakness [1], is a \emph{side channel attack}.

	\bigskip

	We focus on \textbf{timing attacks} for \textbf{software implementations} of AES.

	\bigskip

	We assume the attacker can \emph{easily} capture this timing information.
\end{frame}

\section{History of Timing Attacks}
\begin{frame}
	\frametitle{History of Timing Attacks on Cryptographic Primitives}
	\begin{itemize}
		\item RSA's modular exponentiation ($c \equiv m^e (\mod n)$)
		\begin{itemize}
			\item Square-and-multiply algorithms for $\mathcal{O}(\log_2(n))$ complexity has branch statement whose execution depends directly on $e$.
		\end{itemize}
		\item Branch statements to compute the multiplicative inverse of elements in $GF(2^8)$ [3].
		\item Timing attacks against OpenSSL [4].
		\item Cache hit ratio is predicted to be a fruitful side-channel for launching attacks [5].
	\end{itemize}
\end{frame}

\begin{frame}
	\frametitle{Timing Attacks on AES}
	\begin{itemize}
		\item Rijndael was deemed not susceptible to timing attacks in the AES contest
		\item AES attacks can be based on \emph{statistical} evidence [2].
		\begin{itemize}
			\item Observation: The entire encryption time can be affected by the input bytes $p_i^0 \oplus k_i^0$
			\begin{itemize}
				\item Why? % that value equals $x_i^0, which is the index into the LUT, and if collisions occur then encryption time is affected
			\end{itemize}
			\item Step 1: Capture timing data on a reference and target machine for each value of a particular input byte $p_i^0 \oplus k_i^0 = x_i^0$
			\item Step 2: Perform correlation between reference and target data
			\item Step 3: Heel click.
		\end{itemize}
\vspace{-1em}
\begin{figure}
\centering
\includegraphics[scale = 0.35]{images/bernstein.png}
\end{figure}

		\item Or they can be more targeted:
		\begin{itemize}
			\item Exploit relationships between secret information of the primitive and known data.
			\item This is the approach of Bonneau et al., among others.
		\end{itemize}
	\end{itemize}
\end{frame}

\section{Cache Timing Attacks}
\begin{frame}
	\frametitle{Cache Memory}
\begin{figure}
\centering
\includegraphics[scale = 0.35]{images/aesCacheLines.pdf}
\end{figure}
$\langle x_i \rangle = \langle x_j \rangle$ are the higher bits of the data entry.

\medskip

Data is pulled into cache based on the most-significant bits in its address.
\end{frame}

\begin{frame}
	\frametitle{Cache Collisions}

\begin{columns}
\column{.5\textwidth}
	Let $a$ and $b$ be two memory addresses looked up in memory. Let $\langle a \rangle$ and $\langle b \rangle$ 
	denote the MSBs of $a$ and $b$, respectively. 
	\begin{itemize}
		\item Cache memory is organized into \emph{lines}
		\item Reads on $a$ and $b$ cause a collision if $\langle a \rangle = \langle b \rangle$ (assuming other memory reads
		have not evicted (or invalidated) $a$ or $b$ from the cache. 
		\item If $\langle a \rangle \not= \langle b \rangle$ then a cache collision \emph{might} occur.
		\item We cannot say for certain whether or not the lower LSBs are equivalent...
	\end{itemize}
\column{.5\textwidth}
	\begin{minipage}[c][.6\textheight][c]{\linewidth}
		\begin{figure}[h!]
			\centering
			\includegraphics[scale=0.38]{images/cache_hit_miss.pdf} 
		\end{figure}
	\end{minipage}
\end{columns}
\end{frame}

\begin{frame}
	\frametitle{Cache Collisions Assumption}
	Let $T_E(K, P)$ be the encryption time for a plaintext $P$ using key $K$. 
	Let $\bar{T}_E(K) = \frac{1}{n}\sum_{i = 1}^{n}T_E(K, P_i)$, where $P_i$ is a random plaintext from $\{0|1\}^{128}$.

	\bigskip

	\textbf{Cache-Collision Assumption [1]}. For any pair of table lookups $i, j$, given
a large enough number of \emph{random} AES encryption that use the \emph{same key}, $\bar{T}_E(K)$ will
be lower when $\langle l_i \rangle = \langle l_j \rangle$ than when $\langle l_i \rangle \not= \langle l_j \rangle$

	\bigskip
	
	Note: The table lookup indices must be \emph{independent} for random plaintexts.
\end{frame}

\begin{frame}
	\frametitle{Attacks from Cache Collisions}
	\begin{center}
	That's it! We may now build an attack based on this result.
	\end{center}
\end{frame}

\section{Cache Timing Attacks on AES}
\subsection{LUT-Based Implementations of AES}
\begin{frame}
	\frametitle{LUT-Based Implementations}
	\vspace{-4em}
	\begin{figure}[h!]
		\centering
		\includegraphics[scale=0.5]{images/aesRounds.pdf} 
	\end{figure}
	Let $X^i$ be the state of AES at round $i$. With the exception of $i = 10$, we have:
\begin{align*}
X^{i+1} = \{ & T_0[x_0^i] \oplus T_1[x_5^i] \oplus T_2[x_{10}^i] \oplus T_3[x_{15}^i] \oplus \{k_0^i, k_1^i, k_2^i, k_3^i\}, \\
& T_0[x_4^i] \oplus T_1[x_{9}^i] \oplus T_2[x_{14}^i] \oplus T_3[x_3^i] \oplus \{k_4^i, k_5^i, k_6^i, k_7^i\}, \\
& T_0[x_8^i] \oplus T_1[x_{13}^i] \oplus T_2[x_2^i] \oplus T_3[x_7^i] \oplus \{k_8^i, k_9^i, k_{10}^i, k_{11}^i\}, \\
& T_0[x_{12}^i] \oplus T_1[x_1^i] \oplus T_2[x_6^i] \oplus T_3[x_{11}^i] \oplus \{k_{12}^i, k_{13}^i, k_{14}^i, k_{15}^i\}\}
\end{align*}
\end{frame}

\begin{frame}
	\frametitle{First Round Attack}
\vspace{-4em}
\begin{figure}[h!]
	\centering
	\includegraphics[scale=0.5]{images/aesAttack_first.pdf} 
\end{figure}
\begin{figure}
\centering
\includegraphics[scale = 0.3]{images/first.png}
\end{figure}	
	
%	TODO: show how selection of the six equations works... circle relations on the matrix...
%	0/4, 0/8, 0/12, 4/8, 4/12, 8/12 (six for each T-box), can try to solve from there
%	Limitation: We only know the upper bytes, we cna't figure out the block offset...
\end{frame}

\begin{frame}
	\frametitle{First Round Attack (cont'd)}
\vspace{-4em}
\begin{figure}[h!]
	\centering
	\includegraphics[scale=0.5]{images/aesAttack_first.pdf} 
\end{figure}
\begin{figure}
\centering
\includegraphics[scale = 0.3]{images/firstCircled.png}
\end{figure}	
\end{frame}

\subsection{First Round Attack Idea}
\begin{frame}
	\frametitle{First Round}
	% TODO: image of AES lookup after first round
	\begin{itemize}
		\item First round: $x_i^{0} = p_i \oplus k_i$
		\item With the T-box implementation, $x_0^0$, $x_4^0$, $x_8^0$, and $x_{12}^0$ are 
		used as indices into $T_0$
		\item If we are looking for cache collisions, we must consider input bytes of the same T-box.
	\end{itemize}
	\begin{align*}
		\langle x_i^0 \rangle = \langle x_j^0 \rangle & \Rightarrow \langle p_i \rangle \oplus \langle k_i \rangle = \langle p_j \rangle \oplus \langle k_j \rangle \\
		&  \Rightarrow \langle p_i \rangle \oplus \langle p_j \rangle = \langle k_i \rangle \oplus \langle k_j \rangle 
	\end{align*}
\end{frame}

\begin{frame}[fragile]
	\frametitle{First Round Attack Algorithm}
	\begin{numberedAlg}[FirstRoundAttackSetup($N_s$)]
	\label{eea}
	\begin{algorithmic}[1]
		\State $n \gets 2^8 - 1$, $T \gets$ array$[0 \dots n,1 \dots n, 0 \dots n]$
		\For {$count = 0 \to N_s$}
			\State $p \gets RandomBytes(16)$
			\State $start \gets time()$
			\State $c \gets E_K(p)$ \Comment{Time the encryption}
			\State $end \gets time()$
			\State $tt \gets (start - end)$ \Comment{Increment the time for this particular plaintext difference}
			\For {all $i,j$} \Comment{$i,j$ are input bytes of the \emph{same} T-box}
				\State $T[i,j,\langle p_i \rangle \oplus \langle p_j \rangle] \gets T[i,j,\langle p_i \rangle \oplus \langle p_j \rangle] + tt$
			\EndFor
		\EndFor
		\State $T[i,j,\Delta_{i,j}] \gets T[i,j,\Delta_{i,j}] / N_s$ \Comment{Compute the average time for all $i,j$ pairs}
		\State $\{\Delta_{0,1}', \Delta_{0,1}',\dots,\Delta_{i,j}'\} \gets min(T)$ \Comment{Find the lowest average time for each $i,j$ pair}
		\State $\langle k_{i} \rangle \oplus \langle k_{j} \rangle \gets \Delta_{i,j}'$ for all $i,j$ pairs \Comment{Guess: $\langle k_i\rangle \oplus \langle k_j \rangle = \Delta_{mi,mj}$}
	\end{algorithmic}

	% for i,j with lowest average delta, the key guess is ki \oplus kj = \Delta_{i,j}
	\end{numberedAlg} 
\end{frame}

\begin{frame}
	\frametitle{First Round Limitation}
%	TODO: show how selection of the six equations works... circle relations on the matrix...
%	0/4, 0/8, 0/12, 4/8, 4/12, 8/12 (six for each T-box), can try to solve from there
%	Limitation: We only know the upper bytes, we cna't figure out the block offset...
We only know that $\langle k_0 \rangle \oplus \langle k_4 \rangle = \Delta_{0,4}$, $\langle k_0 \rangle \oplus \langle k_8 \rangle = \Delta_{0,8}$,
$\langle k_0 \rangle \oplus \langle k_{12} \rangle = \Delta_{0,12}$, $\langle k_4 \rangle \oplus \langle k_8 \rangle = \Delta_{4,8}$, 
$\langle k_4 \rangle \oplus \langle k_{12} \rangle = \Delta_{4,12}$, and $\langle k_8 \rangle \oplus \langle k_{12} \rangle = \Delta_{8,12}$.

\bigskip

There exists \textbf{18} other equations we can derive for $T_1$, $T_2$, and $T_3$.

\bigskip

We cannot determine the lower $\log_2$ bits of each key... What to do now?

\end{frame}

\subsection{Last Round Attack}
\begin{frame}
	\frametitle{The Last Round}
	\vspace{-4em}
\begin{figure}
\centering
\includegraphics[scale = 0.5]{images/aesAttack_last.pdf}
\end{figure}
	When $i = 10$, the lookup table is just the S-box $S$. At this point, the ciphertext $C$ is:
\begin{align*}
C = \{ & S[x_0^{10}] \oplus k_0^{10}, S[x_5^{10}] \oplus k_1^{10}, S[x_{10}^{10}] \oplus k_2^{10}, S[x_{15}^{10}] \oplus k_3^{10}, \\
& S[x_4^{10}] \oplus k_5^{10}, S[x_9^{10}] \oplus k_6^{10}, S[x_{14}^{10}] \oplus k_7^{10}, S[x_3^{10}] \oplus k_7^{10}, \\
& S[x_8^{10}] \oplus k_8^{10}, S[x_{13}^{10}] \oplus k_9^{10}, S[x_2^{10}] \oplus k_{10}^{10}, S[x_7^{10}] \oplus k_{11}^{10}, \\
& S[x_{12}^{10}] \oplus k_{12}^{10}, S[x_{1}^{10}] \oplus k_{13}^{10}, S[x_6^{10}] \oplus k_{14}^{10}, S[x_{11}^{10}] \oplus k_{15}^{10}\} 
\end{align*}
\end{frame}

\begin{frame}
	\frametitle{Final Round Collisions}
	Let $x_s$ and $x_t$ be two random bytes in the last round. 
	
	\bigskip
	
	We will always have that $c_i = k_i^{10} \oplus S[x_s]$ and $c_j = k_j^{10} \oplus S[x_t]$. If $x_s = x_t$, then a collision \emph{will usually} occur
	and $c_i = k_i^{10} \oplus \alpha$ and $c_j = k_j^{10} \oplus \alpha$.
	
	\bigskip

	Therefore, $c_i \oplus c_j = k_i^{10} \oplus k_j^{10}$
	
\end{frame}

\begin{frame}
	\frametitle{Final Round Misses}
	
	What if $x_s \not= x_t$?
	
	\bigskip 
	
	$c_i \oplus c_j \not= k_i^{10} \oplus k_j^{10} \Rightarrow$ two different values came out of the LUT!

\end{frame}

\begin{frame}
	\frametitle{S-Box Nonlinearity}
\begin{figure}
\centering
\includegraphics[scale = 0.6]{images/sboxLutDiff.pdf}
\end{figure}
\end{frame}

\begin{frame}
	\frametitle{Final Round Misses (cont'd)}
	\bigskip
	
	The \emph{S-box nonlinearity} means that the difference between $S[x_s]$ and $S[x_t]$ \emph{does not} imply a fixed difference between $x_s$ and $x_t$. 
	
	\bigskip
	
	 If $c_i \oplus c_j \not= k_i^{10} \oplus k_j^{10}$, then $x_s$ and $x_t$ are two \emph{random} values.
\end{frame}

\begin{frame}
	\frametitle{Final Round Attack Algorithm}
	\begin{numberedAlg}[LastRoundAttackSetup($N_s$)]
	\begin{algorithmic}[1]
		\State $n \gets 2^8 - 1$
		\State $diffs \gets [0,\dots,2^n - 1, 0,\dots,2^n - 1]$
		\State $T \gets$ array$[0 \dots 2^n - 1,0 \dots 2^n - 1, 0 \dots 2^n - 1]$
		\For {$count = 0 \to N_s$}
			\State $p \gets RandomBytes(16)$
			\State $start \gets time()$
			\State $c \gets E_K(p)$ \Comment{Time the encryption}
			\State $end \gets time()$
			\State $tt \gets (start - end)$ \Comment{Increment the time for this particular plaintext difference}
			\For {all $i,j$} \Comment{$i,j$ are input bytes of the \emph{same} T-box}
				\State $T[i,j,C_i \oplus C_j] \gets T[i,j,C_i \oplus C_j] + tt$
			\EndFor
		\EndFor
		\State $T[i,j,\Delta_{i,j}] \gets T[i,j,\Delta_{i,j}] / N_s$ 
		\State $\{\Delta_{0,1}', \Delta_{0,1}',\dots,\Delta_{i,j}'\} \gets min(T)$ 
		\State $\langle k_{i} \rangle \oplus \langle k_{j} \rangle \gets \Delta_{i,j}'$ for all $i,j$ pairs \Comment{Guess: $\langle k_i\rangle \oplus \langle k_j \rangle = \Delta_{mi,mj}$}
	\end{algorithmic}
	\end{numberedAlg} 
\end{frame}

\begin{frame}
	\frametitle{Final Round Attack Algorithm (cont'd)}
	With knowledge of $\Delta_{i,j}$ for all $i,j$ such that $\Delta_{i,j} = k_i^{10} \oplus k_j^{10}$ the attacker can now make informed guesses at the key:
	\begin{itemize}
		\item Idea (1): Define a cost function $c(i,j,\Delta_{i,j})$ such that the output of this function is \emph{low} if $\Delta_{i,j}$ is low.
		\item Idea (2): The correct key will yield the lowest value for the cost function when all of the input sums are considered 
		\begin{itemize}
			\item Minimize the sum of all costs for a given key $K$
			\item $C[K] = \sum_{i,j}c(i,j,\Delta_{i,j})$
		\end{itemize}
		\item Goal: Find the correct key guess that minimizes the total cost
		\item Approach: Local optimization search, where each new candidate key guess is obtained by changing one byte of the key.
	\end{itemize}
	
	\bigskip
	
	Thanks to Rijndael's invertible key schedule, $[k^{10}]$ can be reverted back to $[k^{0}]$.
	
	\bigskip
	
	Done.
\end{frame}

\begin{frame}
	\frametitle{Timing Attack Countermeasures}
	\begin{itemize}
		\item Avoid memory access altogether
		\item Use alternative lookup-tables (i.e. non T-box implementation, similar to the fourth)
		\item Data-independent memory access patterns % read all elements of the t_box and only keep the one of interest...
		\item Mask (randomize) data-dependent techniques
		\item Disable cache :-(
		\item Use processor-specific instructions for AES encryption (e.g. Intel's AES ISA)
	\end{itemize}
\end{frame}	

\begin{frame}
	\frametitle{References}
	\begin{itemize}
		\item [1] J. Bonneau, I. Mironov. Cache-collision timing attacks against AES. \emph{Cryptographic Hardware and Embedded Systems CHES 2006}, Springer Berlin Heidelberg, (2006), 201-215.
		\item [2] D. J. Bernstein. Cache-timing attacks on AES. April 2005. 
		{\tt http://cr.yp.to/antiforgery/cachetiming-20050414.pdf}.
		\item [3] F. Koeune, J. J. Quisquater. A timing attack against Rijndael. \emph{Technical Report CG-1999/1}, June 1999.
		\item [4] D. Brumley, D. Boneh. Remote timing attacks are practical. \emph{Computer Networks} \textbf{48(5)} (2005), 701-706.
		\item [5] J. Kelsey, B. Schneier, D. Wagner, C. Hall. Side channel cryptanalysis of product ciphers. \emph{Journal of Computer Security} \textbf{8(2/3)} (2000). 
	\end{itemize}
\end{frame}	

\begin{comment}
\begin{figure}
\centering
\includegraphics[scale = 0.6]{images/sub_layer.jpg}
\end{figure}
\end{comment}

\end{document}
